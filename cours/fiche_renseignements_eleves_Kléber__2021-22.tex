%% -*- mode: latex; coding: utf-8 -*-
%% Start of LaTeX source for default description, in French.
%autotex% Title: Fiche renseignements - MP2I
%autotex% Scale: 0.80
%autotex% PoliceSize: 10pt

\section*{Fiche renseignements -- MP2I}
\emph{Svp, remplissez rapidement cette fiche. Merci d'avance.}

% \hr{}
\subsection*{Informations de contact}
\begin{itemize}
    \item \textbf{Prénom, Nom} :

    \item \textbf{Courriel} :

    \item \textbf{Téléphone} :
    % \hfill{}
    % \emph{(Optionnel\footnote{~Si je suis en retard ou absent un jour, utile pour que je contacte l'un-e d'entre vous. Le mien est \texttt{06 28 41 22 57}.})}

    % \item \textbf{Site(s) web ?}
    % \hfill{}
    % \emph{(Optionnel)}

    \item \textbf{Compte GitLab / GitHub / Bitbucket / autre} ?
    \hfill{}
    \emph{(Optionnel)}

        \vspace*{20pt}
\end{itemize}

% \hr{}
\subsection*{Votre cursus au lycée}
Présentez les options et spécialités que vous avez suivies au lycée :
\begin{itemize}
    % \item Seconde :
    \item Première :
    \item Terminale :
\end{itemize}

% \vspace*{100pt}

% \paragraph{Recherche ?}
% Si vous avez déjà fait des stages de recherche (L3, M1 ou M2) ou une thèse, citez les titres, durées et lieux de vos expériences de recherches svp :
% \vspace*{50pt}

% \hr{}
\subsection*{Quels objectifs et quelle ambition}
% \subsubsection*{Pourquoi faire cette classe préparatoire ?}
\subsubsection*{Qu'est-ce qui vous a attiré dans cette nouvelle classe de MP2I ?}

\vspace*{40pt}

% \hr{}
\subsubsection*{Que comptez-vous retirer de la formation ? (en plus d'une école à la fin)}

\vspace*{20pt}

% \begin{itemize}
%     \item Comptez-vous faire une thèse ensuite / avez-vous déjà une thèse ?
%     \item Comptez-vous demander un poste de professeur-e agrégé-e directement (dans le secondaire) ?
%     \item Comptez-vous demander un poste en classes préparatoires ?
%     \item Bonus : pensez-vous qu'il soit possible d'utiliser votre future agrégation pour enseigner à l'étranger ?
%         % \vspace*{10pt}
%     \item Bonus : si c'est possible, seriez-vous intéressé pour partir un an enseigner à l'étranger l'an prochain ?
%         % \vspace*{10pt}
% \end{itemize}

% \hr{}
\subsection*{Quelques questions \emph{optionnelles} sur votre situation}

% {\small \emph{(Optionnelles)}}
\begin{itemize}
    \item D'où venez-vous ? (ville, région, pays)
    \item Où habitez-vous cette année ? (internat, chambre seule, colocation, logé-e chez quelqu'un de votre entourage, etc.)
    % \item Est-ce votre première année de préparation à l'agrégation ?
    \item Avez-vous un ordinateur personnel portable ou fixe (\emph{entourez votre réponse}) ?
    \item Avez-vous un smartphone ou un téléphone classique ?
    \item Avez-vous accès à Internet chez vous ? Tout le temps ou occasionnellement ?
    \item Quel système(s) d'exploitation utilisez-vous sur votre ordinateur et autres appareils (smartphone etc) ?
    % \item Êtes-vous normalien-ne / boursier-ère ?
    \item Maîtrise de l'anglais : pas du tout, comme De Funès, pas mauvaise, moyenne, bonne, très bonne, comme Shakespeare !
    \item Maîtrise du français : pas du tout (comme ?), moyenne, bonne, comme Zola !
\end{itemize}


% \hr{}
\subsection*{Quelques questions sur votre préparation}
% \paragraph{Oraux de concours}
% \begin{itemize}
%     \item Avez-vous déjà assisté à un oral de concours ? Si oui, quand ?
%     \item Avez-vous déjà lu le programme des différentes matières de la MP2I (maths, info, physique, SI) ? Si oui, quand ?
%     % \item Avez-vous déjà lu la liste des leçons d'informatique / de maths ? Si oui, quand ?
%     % \item Avez-vous déjà commencé à réfléchir au contenu des leçons (plans) ? Si oui, depuis quand ?
%     % \item Avez-vous déjà commencé à travailler des développements (info / maths) ? Si oui, depuis quand ?
%     % \item Comptez-vous rédiger vos plans / développements en \LaTeX{} (ou un autre moyen, sur ordinateur) ?
%     \item Combien de livres universitaires de maths / info possédez-vous ? Et de livres spécifiques à la MP2I ?
%     % \item Pouvez-vous me citer des titres ou auteurs de références (d'agrég) ? (max une douzaine)
%         \vspace*{30pt}
% \end{itemize}

\paragraph{Programmation}
Entourez ou rayez les bonnes ou mauvaises réponses :
\begin{itemize}
    \item Quelle est votre maîtrise de Python ? inexistante, mauvaise, correcte, bonne, très bonne.
    \item Avez-vous déjà programmé sur calculatrice ? Dans quel langage, assembleur, BASIC ou Python ? Quelle calculatrice Texas Instrument, Casio, HP ou Numworks ?
    \item Avec quel environnement avez-vous programmé en Python jusqu'ici ? Pyzo, Spyder, EduPython, jupyter Notebook, ou autre (précisez) ?
    % \item Cela pose-t-il un problème si les exemples de code (développements en leçons, modélisation) que je montrerai et distribuerai\footnote{~Pour l'instant, quelques uns sont sur \texttt{https://github.com/Naereen/notebooks/tree/master/agreg}} sont en Python / en OCaml ?
    \item Connaissez-vous IPython ? Ou les notebooks Jupyter ? Connaissez-vous Emacs ? Et Visual Studio Code ?
    \item Avez-vous déjà codé / développé un projet informatique \emph{pour votre plaisir}\footnote{~Mais si, mais si, certains le font \dots} ? Si oui, quand ? et parlez-en un peu :
        \vspace*{20pt}

    \item Connaissez-vous déjà un peu \verb|OCaml| ? Si oui, quelle maîtrise ?
    \item Connaissez-vous déjà un peu \verb|bash| ? Si oui, quelle maîtrise ?
    \item Connaissez-vous déjà un peu \verb|make| et les \verb|Makefile| ? Si oui, quelle maîtrise ?
    \item Connaissez-vous \LaTeX{} ? Si oui, quelle maîtrise ?
    \item Quel langage de programmation pensez-vous maîtriser (Python, Java, C, OCaml, JavaScript, autres) ?
        \vspace*{20pt}
\end{itemize}


\subsection*{Langues vivantes 1 et 2}
\begin{itemize}
    \item Quelle est votre langue vivante 1 ?
    \item Quelle est votre langue vivante 2 si vous souhaitez la conserver ?
\end{itemize}

\hr{}
% \vfill{}
\subsection*{Remarques particulières et dernières questions}
\begin{itemize}
    \item Comptez-vous faire du sport ?
    \item Si besoin, précisez ici toute remarque supplémentaire que vous jugez importante.
    \item Cela peut notamment être un problème de santé dont l'équipe pédagogique devrait être mise au courant.
\end{itemize}

% \vspace*{40pt}


%% End of LaTeX source.
